\documentclass[]{SangerLibrary/sanger-present}
\usepackage{JML}
\usetikzlibrary{decorations.pathreplacing,calligraphy}
\graphicspath{{Images/}}

\usepackage{fp}
\usepackage{pgfplots}
\usepackage{siunitx}
% \usepackage[usedvipsnames]{xcolor}

\title{\LARGE First Principles of ML\\ {\footnotesize A Peek Inside the 'Black Box' of Machine Learning}}
\author{\large Jack Fraser-Govil}
\institute{The Wellcome Sanger Institute, Hinxton, UK}
\date{\small 15th October 2024}



\renewcommand\vec[1]{\boldsymbol{\mathbf{#1}}}
\begin{document}
	\maketitlepage


	\begin{frame}{Why bother?}

		In a world with dozens of pre-built ML tools....why bother studying the fundamentals?

		\pause In a word...
		\pause\begin{center}
			\LARGE FOLKLORE
		\end{center}
		% \pause Might get you where you need to go...but often leads you astray
	\end{frame}
	\begin{frame}{ML Folklore}
		\begin{itemize}
			\pitem ADAM vs AdaGrad?
			\pitem Softplus vs ReLu vs Leaky ReLu vs Sigmoid?
			\pitem Cross-Entropy vs Least Squares?
			\pitem Validation set magic numbers
			\pitem (Explainability!) 
		\end{itemize}
	\end{frame}

	\begin{frame}{Today's Agenda}
		The aim for today is:

		\begin{itemize}
			\pitem Classic Perceptron
			\pitem Feedforward Networks
			\pitem Non-Linearity
			\pitem Optimisation \& Backpropagation
			\pitem Convolutional Networks*
		\end{itemize}
		(\small * Time dependent!)

		\pause As we progress you will slowly build up your own ML toolkit, built entirely from scratch! 

	\end{frame}

	% \begin{frame}{}
	\newcounter{custompart}
	\setcounter{custompart}{0}
	\newcommand\partFrame[1]
	{
		\stepcounter{custompart}
		\begin{frame}{}
			\begin{center}
				\Wellcome \huge Part \thecustompart

				#1
			\end{center}
		\end{frame}
	}
	% \end{frame}
	
	\begin{frame}{A Warning}
		
		\begin{center}
			{\Large There will be equations.}
			
			{\pause \Large You will need to know what they mean!}

			{\pause \it Please, please, please, ask if you want clarification on the underlying mathematics and theory! That's why you're here today!} 
		\end{center}
	\end{frame}


	\partFrame{The Perceptron}
	
\def\cuteB{-3}
\def\cuteWA{0}
\def\cuteWB{1}
\def\learn{0.01}
\newcommand\cute[5]
	{
		\def\cuteLabel{#5}
		\def\labelText{Not Cute}
		\ifnum\cuteLabel=1
			\def\labelText{Cute}
		\fi
		\node at ({#1},{#2}) {\includegraphics[width=2cm,height=2cm,keepaspectratio=true]{#3}};
		
		% \node at ({#1},{#2}) {#4 \highlight{}};
		\def\test{#4}
		\def\comp{\highlight}




		%%%PERCEPTRON IMPLEMENTED WITHIN IMAGE PLACEMENT
		\ifnum\test=\comp

			\FPeval{dot}{(\a) + (#1*(\b)) + (#2*(\c))}
			\def\cuteGuess{0}
			\pgfmathparse{\dot<0}

			%COMPUTE ASSIGNED VALUE
			\ifnum\pgfmathresult=0
				\def\cuteGuess{1}
			\fi
			\def\guessText{Not Cute}
			\if\cuteGuess1
				\def\guessText{Cute}
			\fi
			
			%DRAW COLOUR
			\def\highcolour{\wrongcolour}
			\ifnum\cuteGuess=\cuteLabel
				\def\highcolour{\rightcolour}
			\else

				%%%UPDATE WEIGHTS IF GUESSED WRONG
				\xdef\updated{1}
				\FPeval{bup}{(\cuteB) + ((\cuteLabel)-(\cuteGuess))*(\learn)}
				\FPeval{waup}{(\cuteWA) + ((\cuteLabel)-(\cuteGuess))*(\learn)*(#1)}
				\FPeval{wbup}{(\cuteWB) + ((\cuteLabel)-(\cuteGuess))*(\learn)*(#2)}
				\xdef\cuteB{\bup}
				\xdef\cuteWA{\waup}
				\xdef\cuteWB{\wbup}
			\fi
			\fill[\highcolour,opacity=0.2] ({#1},{#2}) circle (1.3);
			% \node[red] at ({#1},{#2}) {\a +#1* \b + #2 * \c = \dot};
			
			\node at ({#1},{#2+1.1}) {\tiny Labelled: \labelText, Guess \guessText};
		\else
			% \node at ({#1},{#2-1.1}) {\tiny \labelText};
		\fi
		
	}
% \setbeamercolor{background canvas}{bg=SangerLight}
% \logo{\SangerLogoBlack}
\def\highlight{0}
\def\rightcolour{green!70!black}
\def\wrongcolour{red!70!black}
\def\highcolour{\rightcolour}
\newcommand\animals{
	\cute{0.7}{3.5}{mouse}{1}{1}
	\cute{2}{6}{rabbit}{2}{1}
	\cute{6}{5.7}{dog}{3}{1}
	\cute{4}{5.8}{cat}{4}{1}
	\cute{9}{5.5}{lion}{5}{0}
	\cute{10}{1}{whale}{6}{0}
	\cute{7}{2.7}{warthog}{7}{0}
	\cute{1.7}{1}{molerat}{8}{0}
	\cute{4}{2}{ayeaye}{9}{0}
	\cute{0.1}{0.1}{ant}{10}{0}
}

\newcommand\confusing{
	\cute{2}{6}{rabbit}{1}{1}
	\cute{9}{5.5}{lion}{2}{0}
	\cute{10}{1}{whale}{3}{1}
	\cute{0.1}{0.1}{ant}{4}{0}
}
	\begin{frame}{Basic Decision Making: Defining Cuteness}

		
		\begin{center}
			\def\w{8}
			\begin{tikzpicture}

				\animals
				\draw[->] (0,0)--node[below]{Size} (10,0);
				\draw[->] (0,0)--node[left]{Fur} (0,6);
				\pause \draw[dashed] (0,1)--node[above,rotate=30] {\textbf{Cute}} node[below,rotate=30] {\textbf{Not Cute}} (8,6);

			\end{tikzpicture}
		\end{center}A

	\end{frame}

	
	\begin{frame}{Splitting The Plane}
		In order to split the plane into two parts, we merely need to define a \textit{line}. 
		
		\pause \textbf{Question: } If we have $N$ dimensions ($N=2$), how many parameters do we need to define a line?

		\pause \textbf{Answer: } The answer is $N$ - in 2 dimensions, this is friendly $y = mx + c \to (m,c)$

		\pause \textbf{Question: } Why then do we need $N+1$ dimensions?
	\end{frame}

	\begin{frame}{The Perceptron}
		We need 3 parameters to define out bi-directional line. The Perceptron classifier algorithm is:

		\begin{equation}
			P(\vec{x}) = \begin{cases} 1 & \text{if } \tilde{\vec{x}} \cdot \vec{w}\footnote{The dot/inner product is covered in Section 3.1 in the notes} > 0 \\ 0 &\text{else} 
		\end{cases}
		\end{equation}

		\pause Where 
		$$\tilde{\vec{x}} = \begin{pmatrix}
			1 \\ \vec{x}
		\end{pmatrix}$$
		\pause $\vec{w}$ our the \textbf{weights}. 
	\end{frame}


	\begin{frame}{Exercise 1: Perceptron Classifier}

	\end{frame}
\end{document}

